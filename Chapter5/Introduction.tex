%\section{Introduction}
%\label{sec:chp5-sec1}
%Up to this point, different aspects of automatic classification of melanoma based on dermoscopy images were discussed as well as a \ac{cad} system proposed.
While cross-polarized dermoscopes are used extensively by clinicians and dermatologists, other aspects of polarized imaging in the field of skin lesion analysis remains unexplored. 
The principles of polarization and state-of-the-art methods were discussed in Chap.\,\ref{chp:chapter2}.
In this chapter, the methods proposed by the research community are divided into three categories: (i) partial Stokes polarimetry, (ii) full Stokes polarimetry, and (iii) Mueller polarimetry.
As shown in~\cite{tchvialeva2013polarization}, Mueller and full Stokes polarimetries are difficult to adjust for real-time and in-vivo imaging.
Therefore, we propose a novel dermocope based on partial Stokes polarimetry, which was tested in the Melanoma Unit of the Hospital Clinic of Barcelona.
Using the images obtained with our polarimetry device, their polarization properties were analyzed and an automated \ac{cad} system for the differentiation of melanoma lesions was proposed.

The remainder of this chapter is organized as follows. 
A detailed description of the proposed polarimeter dermoscope is given in Sect.\,\ref{sec:chp5-sec2}.
Then, the quality and characteristics of the images acquired using this system are presented in Sect.\,\ref{sec:chp5-sec3}.
Our dataset and the proposed framework are presented in Sect.\,\ref{sec:chp5-sec4} and Sect.\,\ref{sec:chp5-sec5}, respectively.
Sect.\,\ref{sec:chp5-sec6} shows the experiments along with the results obtained, and finally, conclusions are drawn in Sect.\,\ref{sec:chp5-sec7}.


%Section~\ref{sec:chp5-sec4} presents our proposed framework for analyzing and incorporating the polarized images.
%These images are studied through different experiments which are explained in Sect.\,\ref{sec:chp5-sec5}.







