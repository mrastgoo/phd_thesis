\section{Introduction}
\label{sec:chp3-sec1}
Dermoscopy is a well-known conventional skin screening technique. % in skin lesion analysis
It is used by most dermatologists for screening and analyzing skin lesions.
The main principle of the dermoscope, and more specifically, the polarized dermoscope, was discussed previously in Chapt.~\ref{chp:chapter1}, Sect.~\ref{subsec:derm}.
Due to the importance of an early diagnosis of melanoma, numerous research studies have been dedicated to vision-based automated systems that assist dermatologists in dermoscopy image analysis.
These studies have focused either on specific areas such as lesion delineation, feature extraction and lesion classification, or complete \ac{cad} systems, which include all the aforementioned areas.

In this chapter, we describe our framework for the classification of dermoscopic images in the following order.
First, a summary of work related to the classification of dermoscopy images is presented in Sect.~\ref{sec:chp3-sec2}.
Then in Sect.~\ref{sec:chp3-sec3} and \ref{sec:chp3-sec4}, the material, datasets and the progress of our research are discussed.
The experiments developed throughout the thesis, their results and conclusions are explained in Sect.~\ref{sec:chp3-sec5}.
Finally, we conclude our findings regarding the developed classification framework in Sect.~\ref{sec:chp3-sec7}.

%%% Local Variables: 
%%% mode: latex
%%% TeX-master: "../thesis"
%%% End: 
