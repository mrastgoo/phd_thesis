\section{Related Work}
\label{sec:chp3-sec2}
This section presents a summary of the algorithms for dermoscopic lesion classification proposed over the past decade.
Graphically, this summary is described in Table~\ref{tab:dermoscopy_Feature} and Fig.~\ref{fig:dermoscopy_Sens_PerformanceComparison}.

In 2001, Merler~et al.\,\cite{merler2001tuning} proposed a cost-sensitive boosting approach for the classification of melanoma (M) against benign (B) lesions, where they report \ac{se} and \ac{sp} of 97\% and 54\%, respectively.
They used a dataset called ``MEDS'', which contained 152 lesions, including 42 melanoma.

In another study, Kreutz~et al.\,\cite{kreutz2001automated} presented a classification framework of M~vs.~B lesions, in a dataset of 423 dermoscopic lesions.
In this study, the authors proposed a segmentation approach based on color clustering and a region growing technique. 
Using \acf{mlp} and features such as color statistics, shape and Gabor features, they reported \ac{se} and \ac{sp} of 98.7$\%$ and 76.5$\%$, respectively.

In the same year, Dreiseitl~et al.\,\cite{dreiseitl2001comparison} reported a comparison of different classifiers (kernel \ac{svm}, k-\ac{nn}, \ac{ann}, and \acf{dt}) with a dataset of 1619 lesions, among which 105 lesions were melanoma.
This comparison of M~vs.~B+D (Dysplastic) lesion classification methods reported that the best \ac{se} (92.05$\%$ ) and \ac{sp} (94.97$\%$) was achieved with the kernel \ac{svm}.

Ganster~et al.\,\cite{ganster2001automated} proposed an automated melanoma recognition framework using a very large dataset of 5363 lesions, including 96 melanoma.
The authors proposed segmenting lesions based on a fusion method of several thresholding stages.
Using a k-\ac{nn} classifier and a multiclass classification, the results for \ac{se} and \ac{sp} were 87$\%$ and 92$\%$, respectively.

Rubegni~et al.\,\cite{rubegni2002automated} also proposed an automated diagnosis framework for pigmented lesions, where they used \ac{ann} with shape and color statistics to differentiate M~vs.~B lesions.
Classifying a dataset of 588 lesions, including 217 melanomas, the authors reported \ac{se} and \ac{sp} of 99$\%$ and 92.6$\%$, respectively.

Later, Sboner~et al.\,\cite{sboner2003multiple} proposed an ensemble approach for classification of M~vs.~B lesions.
Their algorithm on a dataset of 152 lesions, containing 42 melanomas, achieved \ac{se} and \ac{sp} of 81$\%$ and 74$\%$, respectively.

Burroni~et al.\,\cite{burroni2005dysplastic} proposed an algorithm for differentiation of M~vs.~D lesions.
In their method, the authors segmented 174 lesions using a Laplacian filter and a zero-crossing algorithm. 
Using the extracted shape, color statistics, and \ac{glcm} features and the stepwise discriminant analysis as the classifier~\cite{burroni2005dysplastic}, they obtained \ac{se} and \ac{sp} of 71$\%$ and 75$\%$, respectively. 

%%% Table 1
\begin{table}[t]
	
	\caption[Summary of the features used in the state of the art]{Summary of the features used in the literature. The references highlighted in \textbf{bold} use \ac{bow} to represent features in a high-dimensional space.}
	\begin{center}
	\begin{threeparttable}
	    \small
	\resizebox{1.0\textwidth}{!}{
	\begin{tabular}{l l r}
	\hline
	\multicolumn{2}{c}{ Feature} & Reference \\
	\hline \\ [-1.5ex]
	\multirow{5}{*}{\textit{Shape}} & Irregularity and compactness ratio  & \cite{kreutz2001automated},\cite{dreiseitl2001comparison},\cite{ganster2001automated},\cite{schaefer2014ensemble},\textbf{\cite{ruela2013role}}\\
	& Fractal geometry & \cite{sboner2003multiple},\cite{kreutz2001automated},\cite{dreiseitl2001comparison},\cite{celebi2007methodological}\\
	& Fourier features & \color{blue}{\cite{ruela2013role}}\\
	& Area and perimeter& \cite{sboner2003multiple},\cite{rubegni2002automated},\cite{schaefer2014ensemble},\cite{burroni2005dysplastic},\cite{celebi2007methodological},\cite{tasoulis2010skin},\cite{ganster2001automated},\cite{capdehourat2011toward},\cite{faal2013improving},\textbf{\cite{ruela2013role}}\\ \\		 
    \multirow{4}{*}{\textit{Color}} &  Alternative color space descriptors  & \cite{capdehourat2009pigmented},\cite{ganster2001automated},\textbf{\cite{barata2013two}}\\
    & Color opponent angle & \textbf{\cite{barata2013two}}\\
    & Color variance and ratio & \cite{sboner2003multiple},\cite{Celebi2015},\cite{kreutz2001automated},\cite{schaefer2014ensemble},\cite{dreiseitl2001comparison},\cite{rubegni2002automated},\cite{burroni2005dysplastic},\cite{celebi2007methodological},\cite{iyatomi2008improved},\cite{tasoulis2010skin},\cite{ganster2001automated},\cite{gilmore2010support},\cite{capdehourat2011toward},\cite{faal2013improving},\textbf{\cite{situ2010modeling}}\\
    & Color quantization & \cite{sboner2003multiple},\cite{celebiautomated},\cite{barata2014color},\textbf{\cite{ruela2013color}},\textbf{\cite{barata2013role}}\\ \\
    \multirow{9}{*}{\textit{Texture}} & Wavelet based descriptor & \cite{garnavi2012computer},\textbf{\cite{situ2008malignant}},\textbf{\cite{situ2010modeling}}\\
    & Local binary pattern & \textbf{\cite{zortea2010automatic}}\\
    & Gabor filter & \cite{kreutz2001automated},\cite{dreiseitl2001comparison},\cite{capdehourat2009pigmented}\\
    & Gray-level co-occurrence matrices &  \cite{celebi2007methodological},\cite{iyatomi2008improved},\cite{Celebi2015},\cite{tasoulis2010skin},\cite{capdehourat2011toward},\cite{faal2013improving},\cite{schaefer2014ensemble},\textbf{\cite{situ2008malignant}}\\
    & Contrast and Entropy & \cite{rubegni2002automated}\\ 
    & Histogram of gradients & \textbf{\cite{barata2013two}}\\
    & SIFT and Color SIFT & \textbf{\cite{situ2010modeling}}\\
    & Difference of Gaussian & \textbf{\cite{barata2013role}}\\
    & Harris / Hessian -Laplace & \textbf{\cite{barata2013role}}\\ 
    [+1.5ex] 
    \hline %\\   
  \end{tabular}
  }
  \end{threeparttable}
\end{center}
\label{tab:dermoscopy_Feature}
\end{table}

Similar features were used later in \cite{celebi2007methodological}.
In their framework, the authors proposed segmenting lesions using a region growing and merging algorithm with a color similarity constraint.
Their experiments for classification of M~vs.~B+D lesions using kernel \ac{svm}, with a dataset of 564 lesions containing 88 melanomas, achieved \ac{se} and \ac{sp} of 92.34$\%$ and 93.33$\%$, respectively.

Iyatomi~et al.\,\cite{iyatomi2008improved} proposed a \ac{cad} system for recognition of M~vs.~B+D lesions.

The authors segmented lesions using a region growing algorithm and classified some shape, color and texture (\ac{glcm}) features, which resemble the ``ABCD'' characteristics, with an \ac{ann} classifier.
They reported that with a large dataset of 1258 lesions including 198 melanomas their algorithm achieved \ac{se} and \ac{sp} of 85.9$\%$ and 86$\%$, respectively.

Situ~et al.\,\cite{situ2008malignant} proposed representing image features using the \ac{bow} approach.
Classifying wavelet and Gabor texture features using the \ac{bow} representation and an \ac{svm} classifier, the authors reported \ac{se} and \ac{sp} of 88.8$\%$ and 69.26$\%$, respectively.
This study was dedicated to the differentiation of M~vs.~B+D lesions with a dataset of 100 lesions, containing 30 melanomas.
In a later study~\cite{situ2010modeling}, the authors proposed a different segmentation method based on the graph cut algorithm, and, using the same classification framework, they extracted color, \ac{sift}, and wavelet features instead.
As a result, with the same dataset, they achieved \ac{se} of 83$\%$ and \ac{sp} of 80.93$\%$. 

Another computerized framework was proposed by Capdehourat~et al.\,\cite{capdehourat2009pigmented} for the classification of M~vs.~B+D lesions with a dataset of 433 lesions, among which 80 lesions were melanomas.
The authors used the Dullrazor\textsuperscript{\textregistered}~\cite{Lee1997} algorithm for hair removal and proposed segmenting lesions using color thresholding based on Otsu's approach.
Using color statistics and texture features from \ac{glcm} and Gabor filters, and the \ac{adb} classifier, they achieved \ac{se} and \ac{sp} of 95$\%$ and 91.25$\%$, respectively.
They also proposed using the \ac{smote} balancing algorithm to deal with imbalanced data distributions.
Using the same pre-processing and classification framework on a larger dataset (544 lesions containing 111 melanomas), they later achieved \ac{se} and \ac{sp} of 90$\%$ and 77$\%$, respectively~\cite{capdehourat2011toward}. 

A combination of \ac{svm} and \ac{bow} representation was also used in \cite{zortea2010automatic} to differentiate M~vs.~B lesions in a dataset of 164 lesions (including 80 melanomas).
Using the \ac{bow} representation of \ac{lbp} features, the authors eliminated the segmentation stage and achieved \ac{se} and \ac{sp} of 73$\%$ and 73$\%$, respectively.
Later, using a larger dataset (206 lesions containing 37 melanomas), the authors proposed a framework for classification of M~vs.~B+D based on shape, color and texture features mimicking the ``ABCD'' characteristics~\cite{zortea2014performance}.
In this work, using feature selection techniques and the \acf{qda} classifier, their best results reached \ac{se} of 86$\%$ and \ac{sp} of 52$\%$. 

The \acl{svm} classifier was also used in \cite{gilmore2010support} for automatic recognition of M~vs.~D lesions.
In this work, the authors proposed extracting color statistics and shape features from the entire image without considering the segmentation step.
They later used a \ac{pca} to reduce the feature dimensions and achieved \ac{se} and \ac{sp} of 89$\%$ and 64$\%$.

In other work, Garnavi~et al.\,\cite{garnavi2012computer} proposed a \ac{cad} system for classification of M~vs.~B lesions.
In this study, the author compared manual and adaptive thresholding approaches for the segmentation of lesions.
As for feature extraction, they combined wavelet package features with shape statistics.
Their paper also provided a comparison between \ac{svm}, \ac{rf}, \ac{lmt}, and hidden \ac{nb} classifiers, where the best \ac{acc} of 88.30$\%$ was achieved with the \ac{rf} classifier.

Abbas~et al.\,\cite{abbas2012computer} proposed another \ac{cad} system for the classification of M~vs.~B+D lesions.
In this framework, the authors proposed their own hair removal~\cite{Abbas2011a} and lesion segmentation~\cite{Abbas2012} algorithms.
Using the \ac{svm} classifier and \ac{lbp}, wavelet, and color statistics as features, with a relatively small dataset of 120 lesions (containing 20 melanomas) they reported \ac{se} and \ac{sp} of 91.46$\%$ and 94.14$\%$, respectively.
Using the same framework, they later reported their results with a different dataset of 120 lesions, which contained 60 melanomas~\cite{abbas2013melanoma}.
With this dataset, the \ac{se} and \ac{sp} were at 88.2$\%$ and 91.3$\%$, respectively.

Barata~et al.\,\cite{barata2013two,barata2013role} and Ruela~et al.\,\cite{ruela2013role,ruela2013color} recently presented their research on different subsets of the PH$^{2}$ dataset~\cite{mendoncca2013ph}.
In this work, the authors compared the role of shape and color descriptors~\cite{ruela2013role,ruela2013color} for the detection of M~vs.~B+D lesions using the \ac{adb} classifier.
In \cite{barata2013two,barata2013towards}, they proposed using the \ac{bow} representation of color and gradient features.
Their results comparing the \ac{adb}, kernel \ac{svm}, and k-\ac{nn} classifiers indicated that the combination of the \ac{bow} representation and k-\ac{nn} classifier achieved the best performance with \ac{se} and \ac{sp} of 100$\%$ and 75$\%$, respectively.

Recent studies on automated melanoma recognition were presented in \cite{schaefer2014ensemble,Celebi2015}.
Schaefer~et al.~\cite{schaefer2014ensemble} proposed a M~vs.~B+D classification framework, using a dataset of 564 lesions, among which 88 lesions were melanoma.
The authors chose the \ac{smote} sampling to avoid the imbalance problem and used ensemble approaches to perform the classification.
Using shape, color and \ac{glcm} features, they achieved \ac{se} and \ac{sp} of 93.76$\%$ and 93.84$\%$, respectively.
Shimizu~et al.~\cite{Celebi2015} tackled a multiclass classification problem between melanoma, basal cell carcinoma, seborrheic keratosis, and dysplactic nevus.
They used an ensemble approach for classifying each group with a dataset of 964 lesions with 109 melanomas.
In their study, using color and \ac{glcm} features \ac{se} of 90.48$\%$ for melanoma lesions was achieved.
%\clearpage
\definecolor{autoGuided}{rgb}{ 0.3765    0.7294    0.9412}
\newcommand{\autoGuidedColor}{(light-Blue)}
\definecolor{fullyAuto}{rgb}{ 0.0941    0.3843    0.6627}
\newcommand{\fullyAutoColor}{(dark-blue)}
\definecolor{semiAuto}{rgb}{ 0.0784    0.5059    0.1686}
\newcommand{\semiAutoColor}{(light-green)}
\definecolor{fullyGuided}{rgb}{ 0.4275    0.6902    0.3176}
\newcommand{\fullyGuidedColor}{(dark-green)}
\begin{figure}
\centering
%\subfloat[]{       
{
\begin{tikzpicture} [scale=.99,every node/.style={scale=0.9}]  %[scale=.68]

\def\labels{
{\color{dgreen}\cite{sboner2003multiple}}, 
{\color{dgreen}\cite{merler2001tuning}}, 
{\color{dgreen}\cite{kreutz2001automated}},  
{\color{dgreen}\cite{rubegni2002automated}}, 
{\color{dgreen}\cite{situ2010modeling}}, 
{\color{dgreen}\cite{zortea2010automatic}},
{\color{dgreen}\cite{zortea2014performance}}, 
{\color{dgreen}\cite{burroni2005dysplastic}}, 
{\color{dgreen}\cite{gilmore2010support}}, 
{\color{dgreen}\cite{abbas2013melanoma}}, 
{\color{dgreen}\cite{barata2013two}}, 
{\color{dgreen}\cite{barata2013two}}, 
{\color{dgreen}\cite{ruela2013color}},
{\color{dgreen}\cite{ruela2013role}},
{\color{dgreen}\cite{barata2013role}},
{\color{dgreen}\cite{schaefer2014ensemble}}, 
{\color{dgreen}\cite{ganster2001automated}},
{\color{dgreen}\cite{celebi2007methodological}}, 
{\color{dgreen}\cite{situ2008malignant}}, 
{\color{dgreen}\cite{iyatomi2008improved}}, 
{\color{dgreen}\cite{capdehourat2009pigmented}}, 
{\color{dgreen}\cite{capdehourat2011toward}}, 
{\color{dgreen}\cite{abbas2012computer}}, 
{\color{dgreen}\cite{dreiseitl2001comparison}},
{\color{dgreen}\cite{faal2013improving}}
 }


%\cite{Celebi2015}, SE 90.48, multiclass, 105/964, Segmentation Color thresholding, color, GLCM %90.48, %82.51 but its four class classification,

\def\reward{81,97,98.7,99,83,73,86,71,89,88.2,96,100,96,92,98,93.76,87,92.34,88.8,85.9,95,90,91.64,92.05,83.53}
\def\spec{74,54,76.5,72.5,80.93,73,52,75,64,91.3,80,75,83,78,86,93.84,92,93.33,69.26,86,91.25,77,94.14,94.97,67.73}
\def\dbSize{42/152,42/152,na/423,217/588,30/100,80/164,37/206,38/174,101/199,60/120,25/176,25/176,24/169,24/169,25/167,88/564,96/5363,88/564,30/100,198/1258,80/433,111/544,20/180,105/1619,44/436}
\def\dbClass{2,2,3,4,1,2,2,2,2,2,2,2,2,2,2,4,5,4,1,5,3,4,2,5,3}
	
\def\cZoom{4.5} 
\def\percentageLabelAngle{10}
\def\nbeams{25}
\pgfmathsetmacro\beamAngle{(360/\nbeams)}
\pgfmathsetmacro\halfAngle{(180/\nbeams)}

\pgfmathsetmacro\globalRotation{\halfAngle}

\foreach \n  [count=\ni] in \labels
{
\pgfmathsetmacro\cAngle{{(\ni*(360/\nbeams))+\globalRotation}}
\draw	(\cAngle:{\cZoom*1.00})  node[fill=white] {\n};
\draw [thin] (0,0) -- (\cAngle:{\cZoom*0.9}) ;

}

% draw the % rings 
\foreach \x in {12.5,25, ...,100} 
\draw [thin,color=gray!50] (0,0) circle [radius={\cZoom*\x/110}];

\foreach \x in {50,75,100}
{ 
     \draw [thin,color=black!50] (0,0) circle [radius={\cZoom/110*\x}];
     \foreach \a in {0, 180} \draw ({\percentageLabelAngle+\a}:{\cZoom*0.01*\x}) node  [inner sep=0pt,outer sep=0pt,fill=white,font=\fontsize{3}{3.5}\selectfont]{$\x\%$};
}


% draw the path of the percentages
\def\aux{{\reward}}

\pgfmathsetmacro\origin{\aux[\nbeams-1]} 
\draw [blue, thick] (\globalRotation:{\cZoom*\origin/110}) \foreach \n  [count=\ni] in \reward { -- ({(\ni*(360/\nbeams))+\globalRotation}:{\cZoom*\n/110}) } ;


\def\auxx{{\spec}}
\pgfmathsetmacro\origin{\auxx[\nbeams-1]} 
\draw [black, thick] (\globalRotation:{\cZoom*\origin/110}) \foreach \n  [count=\ni] in \spec { -- ({(\ni*(360/\nbeams))+\globalRotation}:{\cZoom*\n/110}) };

% label all the percentags
\foreach \n [count=\ni] in \dbSize 
{
	\pgfmathsetmacro\cAngle{{(\ni*(360/\nbeams))+\globalRotation}}
	\pgfmathsetmacro\nreward{\aux[\ni-1]}
	\pgfmathsetmacro\nspec{\auxx[\ni-1]}
	\draw (\cAngle:{\cZoom*1.36}) node[align=center] {{\color{blue}\nreward $\%$ } \\{ \color{black}\nspec $\%$} \\{\color{red}\n}};
} ;


% draw the database rose
\def\dbScale{\9}
\foreach \n [count=\ni] in \dbClass
\filldraw[fill=red!20!white, draw=red!50!black]
(0,0) -- ({\ni*(360/\nbeams)-\halfAngle+\globalRotation}:{\cZoom*\n/9}) arc ({\ni*(360/\nbeams)-\halfAngle+\globalRotation}:{\ni*(360/\nbeams)+\halfAngle+\globalRotation}:{\cZoom*\n/9}) -- cycle;
\foreach \x in {1,2,3,4,5}
\draw [thin,color=red!50!black,dashed] (0,0) circle [radius={\cZoom*\x/9}];
   	
%%% draw the domain of each class 
  \def\puta{	7/0/{M vs. B},	
  			2/7/{M vs. D},
  			16/9/{M vs. D+B}}
\foreach \numElm/\contadorQueNoSeCalcular/\name [count=\ni] in \puta
 {

 	\pgfmathsetmacro\initialAngle{(\contadorQueNoSeCalcular*\beamAngle)+\halfAngle+\globalRotation}
 	\pgfmathsetmacro\finalAngle  {((\numElm+\contadorQueNoSeCalcular)*\beamAngle)+\halfAngle+\globalRotation}
	\pgfmathsetmacro\l  {\cZoom*1.5+.3pt}
	\draw (\initialAngle:{\cZoom*1.6}) -- (\initialAngle:{\cZoom*1.1});
	\draw [ |<->|,>=latex] (\initialAngle:\l) arc (\initialAngle:\finalAngle:\l) ;    									 
	\pgfmathsetmacro\r  {\cZoom*1.5+.45pt}
    	{\draw [decoration={text along path,  text={\name},text align={center}},decorate] (\finalAngle:\r) arc (\finalAngle:\initialAngle:\r);}
  
  
  
  }  
\end{tikzpicture}
}

\caption[Summary of the literature]{\small{Summary of the classification performances of the methods reviewed from the dermoscopic imaging literature.
The main results of the authors (references are presented in {\color{dgreen} green}) are illustrated in {\color{blue}blue} and black as sensitivity and specificity, respectively, while the size of the dataset is represented in {\color{red}red} (cf., number of melanoma lesions over the total number of lesions).
A comparison of the size of the datasets is also presented in the middle of the graph, which contains five groups.
We categorized datasets with less than 100 lesions as group~1, datasets with more than 100 and less than 200 lesions as group~2, dataset with less than 500 lesions as group~3, any datasets with over 500 lesions and less than 1000 lesions as group~4 and finally datasets with over 1000 lesions as group~5.
In this graph, $M$ vs. $D$ and $M$ vs. $B$ indicate that the dataset contains only malignant/dysplastic or melanoma/benign lesions, respectively, while $M$ vs. $D+B$ indicates that melanoma lesions were classified against benign and dysplastic lesions.}} 
\label{fig:dermoscopy_Sens_PerformanceComparison}
\end{figure}


All the studies described above are summarized in Fig.~\ref{fig:dermoscopy_Sens_PerformanceComparison} and Table~\ref{tab:dermoscopy_Feature}.
%and Table.~\ref{tab:dermoscopy_Classifiers}.
Figure.~\ref{fig:dermoscopy_Sens_PerformanceComparison} categorizes the methods in terms of the dataset size, classification scope (M~vs.~B, M~vs.~B~+~D, and M~vs.~D) and results achieved (\ac{se} and \ac{sp}).
In this figure, a categorization of different datasets is presented together with the ratio of melanoma samples over the total size of the dataset.
The datasets with less than 100 lesions are categorized as group~1, group~2 contains datasets with more than 100 and less than 200 lesions; datasets with more than 200 and less than 500 lesions are in group~3, while groups~4 and 5 contain datasets from 500 up to 1000 and over 1000 lesions, respectively.
%belong to group~3 and dataset with over 500 lesions and less than 1000 lesions are categorized as group~4 and finally any dataset over 1000 lesion is categorized as group~5.
Table~\ref{tab:dermoscopy_Feature} categorizes the research studies in terms of their choice of features.
In this table, the references that chose a \ac{bow} representation of the extracted features are highlighted in \textbf{bold}.
%In this table, the references which opted for the \ac{bow} representation of the extracted features are highlighted in \textbf{bold}.
%% Table 1
\begin{table}[t]
	
	\caption[Summary of the features used in the state of the art]{Summary of the features used in the literature. The references highlighted in \textbf{bold} use \ac{bow} to represent features in a high-dimensional space.}
	\begin{center}
	\begin{threeparttable}
	    \small
	\resizebox{1.0\textwidth}{!}{
	\begin{tabular}{l l r}
	\hline
	\multicolumn{2}{c}{ Feature} & Reference \\
	\hline \\ [-1.5ex]
	\multirow{5}{*}{\textit{Shape}} & Irregularity and compactness ratio  & \cite{kreutz2001automated},\cite{dreiseitl2001comparison},\cite{ganster2001automated},\cite{schaefer2014ensemble},\textbf{\cite{ruela2013role}}\\
	& Fractal geometry & \cite{sboner2003multiple},\cite{kreutz2001automated},\cite{dreiseitl2001comparison},\cite{celebi2007methodological}\\
	& Fourier features & \color{blue}{\cite{ruela2013role}}\\
	& Area and perimeter& \cite{sboner2003multiple},\cite{rubegni2002automated},\cite{schaefer2014ensemble},\cite{burroni2005dysplastic},\cite{celebi2007methodological},\cite{tasoulis2010skin},\cite{ganster2001automated},\cite{capdehourat2011toward},\cite{faal2013improving},\textbf{\cite{ruela2013role}}\\ \\		 
    \multirow{4}{*}{\textit{Color}} &  Alternative color space descriptors  & \cite{capdehourat2009pigmented},\cite{ganster2001automated},\textbf{\cite{barata2013two}}\\
    & Color opponent angle & \textbf{\cite{barata2013two}}\\
    & Color variance and ratio & \cite{sboner2003multiple},\cite{Celebi2015},\cite{kreutz2001automated},\cite{schaefer2014ensemble},\cite{dreiseitl2001comparison},\cite{rubegni2002automated},\cite{burroni2005dysplastic},\cite{celebi2007methodological},\cite{iyatomi2008improved},\cite{tasoulis2010skin},\cite{ganster2001automated},\cite{gilmore2010support},\cite{capdehourat2011toward},\cite{faal2013improving},\textbf{\cite{situ2010modeling}}\\
    & Color quantization & \cite{sboner2003multiple},\cite{celebiautomated},\cite{barata2014color},\textbf{\cite{ruela2013color}},\textbf{\cite{barata2013role}}\\ \\
    \multirow{9}{*}{\textit{Texture}} & Wavelet based descriptor & \cite{garnavi2012computer},\textbf{\cite{situ2008malignant}},\textbf{\cite{situ2010modeling}}\\
    & Local binary pattern & \textbf{\cite{zortea2010automatic}}\\
    & Gabor filter & \cite{kreutz2001automated},\cite{dreiseitl2001comparison},\cite{capdehourat2009pigmented}\\
    & Gray-level co-occurrence matrices &  \cite{celebi2007methodological},\cite{iyatomi2008improved},\cite{Celebi2015},\cite{tasoulis2010skin},\cite{capdehourat2011toward},\cite{faal2013improving},\cite{schaefer2014ensemble},\textbf{\cite{situ2008malignant}}\\
    & Contrast and Entropy & \cite{rubegni2002automated}\\ 
    & Histogram of gradients & \textbf{\cite{barata2013two}}\\
    & SIFT and Color SIFT & \textbf{\cite{situ2010modeling}}\\
    & Difference of Gaussian & \textbf{\cite{barata2013role}}\\
    & Harris / Hessian -Laplace & \textbf{\cite{barata2013role}}\\ 
    [+1.5ex] 
    \hline %\\   
  \end{tabular}
  }
  \end{threeparttable}
\end{center}
\label{tab:dermoscopy_Feature}
\end{table}
%Table~\ref{tab:dermoscopy_classifiers} categorized the 

Based on our study of the state-of-the-art lesion classification methods, we made several important findings listed in the following: 
\textbf{
\begin{enumerate}[i]
\item Color, shape, and often \ac{glcm} features are used more extensively compared to others.
Although these features are applied to represent characteristics similar to the ``ABCD'' rule, the lack of attention to other well-known pattern recognition features is evident.
%Features such as color, shape, and often \ac{glcm} are used extensively compared to others. 
%Although, these features are used to represent similar characteristics to ``ABCD'' rule, the lack of using well-known pattern recognition features is evident.
\item Different machine learning techniques are used to perform the classification. However, the application of ensemble classifiers was limited to only few studies.
\item The majority of the datasets employed are imbalanced, in some cases to a great degree~\cite{ganster2001automated}.
However, balancing techniques are rarely used to reduce the bias of the classifier.
%Normally the used datasets are imbalanced and in some cases highly imbalanced~\cite{ganster2001automated}.
%However balancing techniques are rarely used to reduce the bias of the classifier.
\item As it is highlighted in Fig.~\ref{fig:dermoscopy_Sens_PerformanceComparison}, few methods were proposed for the classification of M~.vs~D lesions.
Classification of M against D lesions is more challenging due to the characteristic similarities between these types of lesions.
This is a challenging task for both dermatologists and automated algorithms in comparison with differentiation between M~.vs~B lesions, which is more straight forward for dermatologists.
\item Finally, although some methods achieved very good results, it is impossible to offer a fair comparison of the above frameworks because their performance is reported using different datasets.
Consequently, it is not possible to draw definitive conclusions about their performance in classification of melanoma lesions.
%for the performance of machine learning techniques in classification of melanoma lesions.
\end{enumerate}}
Keeping these conclusions in mind, we propose our \ac{cad} system for the classification of melanoma lesions, which is discussed in great detail in the rest of this chapter.



%%%%%%%%%% Dermoscopy 
%\cite{Celebi2015}, SE 90.48, multiclass, 105/964, Segmentation Color thresholding, color, GLCM
%\cite{schaefer2014ensemble}, SE 93.76, SP 93.84, 88/564, balancing, Segmentation JSEG, ensemble NN, shape, color, GLCM 
%\cite{mete2014optimal}, SVM, 43/90, acc 92.2, shape, color based on ABCD
%\cite{zortea2014performance}, LDA, QDA, SE 86, SP 52, D = 37/206,segmentation = machine learning, color, shape, texture LBP.
%\cite{abbas2013melanoma},SVM,  SE 88.2, SP 91.3, color features, segmentation SETA, hair removal = derivative of gaussian, morphological function, fast marching, SFFS, 60/120
%\cite{barata2013two} SE 96 SP 80 , 25/176 , Knn, AdB SVM  global Color, Gradient 
%\cite{barata2013two} SE 100 SP 75 , 25/176 , Knn, AdB SVM local , BOF, color, Gradient
%\cite{ruela2013color} SE (96), SP(83), 24/169, AdB, Color
%\cite{ruela2013role} SE (92), SP (78), 24/169, AdB, 
%\cite{barata2013role}  SE 98 SP 86, 25/167, BOW 
%\cite{abbas2012computer} SE 91.64  SP 94.14, 20/180, SVM
%\cite{capdehourat2011toward} , SE 90 SP  77 , 111/544, SVM  AdB

%\cite{garnavi2010classification} Acc  88.24 , 96/205 , RF SVM LMT 
%\cite{tasoulis2010skin} Clustering , 69/3631 , DePDDA 
%\cite{situ2010modeling} SE 83 SP 80.93 , 30/100 , SVM
%\cite{zortea2010automatic}  SE 73 , SP 73 , 80/164 , SVM 
%\cite{gilmore2010support} SE 89.0 SP 64.0 , 101/109 , SVM 

%\cite{capdehourat2009pigmented} SE 95 SP 91.25, 80/433 , AdB, SVM 
%\cite{situ2008malignant} SE 88.8 SP 69.26, 30/100 , SVM  NB
%\cite{surowka2008supervised}  SE 89.2 SP 90 , 19/39 , ANN and SVM 
%\cite{iyatomi2008improved} SE 85.9  SP 86.0 , 198/1258 , ANN 
%\cite{chiem2007novel} , Acc 95.5 , NA / 100, SVM 
%\cite{celebi2007methodological} SE 92.34 SP  93.33 , 88/564 , SVM RBF 
%\cite{burroni2005dysplastic} SE 71 SP 75, 38/174 , stepwise discriminant analysis 
%\cite{sboner2003multiple} SE 81.0 SP 74.0, 42 / 152, LDA, KNN, DT  combination of three classifiers 
%\cite{rubegni2002automated} SE 99 SP 72.5 , 217/588, ANN 

%\cite{merler2001tuning} SE 97.0  SP 54.0, 42/152 , AdB
%\cite{kreutz2001automated}  SE 98.7 , SP 76.5, NA/423, ANN 
%\cite{dreiseitl2001comparison} SE 92.05 , SP 94.97 , 105/1619, SVM RBF 
%\cite{ganster2001automated} SE 87 SP 92, 96/5363, KNN 


%%% Local Variables: 
%%% mode: latex
%%% TeX-master: "../thesis"
%%% End: 
