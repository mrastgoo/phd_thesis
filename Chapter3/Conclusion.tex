\section{Conclusion}
\label{sec:chp3-sec7}
In this chapter, we presented our automated recognition framework for the classification of melanoma lesions in dermoscopy images.
The proposed framework consists of six main steps: image preprocessing, image mapping, feature extraction, feature representation, data balancing, and classification.
For each of these steps, we made the following contributions:
\begin{itemize}
	\item Proposed a hair removal algorithm based on morphological operations and exemplar-based inpainting.
	\item Developed a segmentation method based on a fusion of \ac{pdf}-based, level-set, and \ac{fcm} algorithms.
	\item Considered different texture and color descriptors beyond the clinical characteristics of the ``ABCD'' rule.
	\item Compared global and local feature extraction methods (mapping). The latter was implemented using a sliding window and dividing each sample into small patches.
	\item Applied the \ac{bow} and \ac{scf} techniques to reduce the complexity and dimensionality of the locally mapped features.
	\item Addressed the class imbalance problem by adapting the \acl{dos} approach to pigmented skin lesions and comparing it with the state-of-the-art over/under-sampling techniques in the feature space.
	\item Compared the performance of different learners and ensembles, such as \ac{svm}, \ac{gb}, \ac{rf} and a weighted combination.
\end{itemize}
%A hair removal algorithm based on morphological operations and exemplar-based inpainting was proposed to detect hairs and repair missing parts of the image.
%We also presented a segmentation method based on a fusion of \ac{pdf}-based, level-set, and \ac{fcm} algorithms.
%We considered different texture and color descriptors beyond the clinical characteristics of the ``ABCD'' rule.
%It was proposed to extract the selected features per element or elements within each sample (global and local mapping, respectively).
%The later is achieved through sliding window and dividing each sample to small patches.
%In order to reduce the complexity and dimensionality of locally mapped features, we proposed to use \ac{bow} or \ac{scf} techniques.
%We also consider how to deal with a common problem of class imbalance.
%In this regard, first we proposed a \acl{dos} approach, suited for pigmented lesions.
%Later we compared our \ac{dos} approach with the state of the art over-under sampling techniques in feature space.
%Finally we compared the performance of different learners and ensembles, such as \ac{svm}, \ac{gb}, \ac{rf} and weighted combination. 

The proposed framework was tested with respect to its individual steps in 5 different experiments.
Experiments~\#1 and \#2 were performed with the Vienna dataset, while the rest were carried out with the PH$^{2}$ dataset.
The results obtained highlighted the superiority of the texture features such as Gabor filters, \ac{clbp}, \ac{hog}, and \ac{sift} as compared to the state-of-the-art \ac{glcm} method.
They also showed that the combination of color and texture features leads to a better performance.
As to the global and local mapping results, we believe it to be highly dependant on the dataset, because in some cases, global feature extraction performs equal or superior to local extraction.
%Testing the framework on two different datasets showed that the discriminative power of the color features are dataset driven and are subject to illumination, device and environment conditions.
%In general terms maintaining discriminative and consistent color features with respect to different datasets or even in one dataset is not straightforward and additional color calibration is required.

The importance of a balanced training, as well as the advantage of balance strategies in the feature space, was demonstrated in different experiments.
In particular, it was shown that the \ac{us} methods are more suited for melanoma recognition than \ac{os}.
In terms of classifiers, the benefit and capability of an \ac{rf} ensemble was shown throughout different experiments.
In general, random forests is a less parameterized classifier, computationally less expensive, requiring minimum user interactions and obtaining the highest results. 
Finally, in Experiment~\#5, we tested the performance of \acl{scf}.
This experiment showed that by using the simplest features (R, G, and B intensities), and without prior segmentation, it is possible to achieve acceptable results.
The highest performance was achieved on the PH$^{2}$ dataset using \ac{sift} features, a sparsity level of 2, and 800 atoms.


Our proposed framework is offline and requires learning stage.
However, depending on the configuration setup, the time required varies.
For instance, high-level representations using dictionary learning such as \ac{bow} or \ac{scf} require more time than low-level representation and, among the classifiers, the \ac{svm} requires more training time, since a grid-search method is used to optimize its parameters.

In line with our objective, the degree of robustness and generality of the proposed framework with respect to different datasets has been questioned.
Although no test has been performed to confirm this aspect, we expect the use of ensembles such as \ac{adb} and \ac{rf}, along with balancing techniques and pruning approaches~\cite{rastgoo2012pruning} will be the solution to this problem.
\subsection{Future Work}
As an avenue for future research, we would like to explore several aspects of our framework in more detail, including:
\begin{enumerate}[i]
\item An evaluation of the hair removal algorithm using annotated data. 
\item Extension and further studies of the segmentation algorithm.
\item A comparison of sparse learned dictionaries with the \acl{bow} model.
\item A more detailed examination of balancing strategies.
\item A comparison of the tools developed with a larger dermoscopy dataset.
\item An evaluation of the degree of robustness and generality of the proposed framework while dealing with data streams over time.
\item Use of the recent techniques such as deep-learning for the classification.
\end{enumerate}

%1 ) Features such as color, shape, and often GLCM are used extensively compare to others.
%Although, this features are used to represent similar characteristics to “ABCD” rule, the
%lack of using well-known pattern recognition features is evident.
%2) Different machine learning techniques are used to perform classification. However the choice
%of ensemble was limited to few studies
%3) Normally the used datasets are imbalanced and in some cases highly imbalanced [75]. How-
%ever balancing techniques are rarely used to reduce the bias of the classifier.
%4) As it is highlighted in Fig. 3.1, few methods were proposed for classification of M .vs D
%lesions. Classification of M against D lesions is more challenging due to the characteristic
%similarities between the lesions. This is a challenging task for both dermatologists and
%automated algorithms in compare to differentiation of M .vs B lesions, which is more straight
%forward for the dermatologists.
%5) Finally, although some methods achieved very good results, it is impossible to offer a fair
%comparison among the aforementioned methods, since their performance are reported using
%different datasets. Subsequently it is not possible to draw a conclusion for the performance
%of machine learning techniques in classification of melanoma lesions.
%
%Keeping these conclusions in mind, we proposed our CAD system for classification of melanoma
%lesions, which we discuss in great details in the following of the chapter.
%


%%% Local Variables: 
%%% mode: latex
%%% TeX-master: "../thesis"
%%% End: 
