%\tikzstyle{decision} = [diamond, draw, fill=blue!20, 
%    text width=4.5em, text badly centered, node distance=3cm, inner sep=0pt]

\tikzstyle{block} = [rectangle, draw, fill= blue!30,  
   text width=7em, text centered, rounded corners, minimum height=4em , minimum width = 7em]
\tikzstyle{line} = [draw, -latex']
\tikzstyle{block2} = [rectangle, draw, fill=white!20,
    text width=7em, text centered, rounded corners, minimum height=4em, minimum width = 7em]
\tikzstyle{block3} = [rectangle, draw, fill=blue!30,
    text width=7em, text centered, rounded corners, minimum height=3em , minimum width = 7em]
\def\blockdist{1}
\def\edgedist{1.5}
\begin{figure}[H] 
\begin{center}   
\begin{tikzpicture}[node distance = 1cm,scale=0.7, every node/.style={scale=0.7}]
    % Place nodes
    % -- input images 
    \node [block2] (input) {Dermoscopy images};
    \path (input.north|- input.east)+(0.1,+1.1\blockdist) node (g) {};
    % -- preprocessing block
    \node [block, right of= g , node distance = 3.9cm](HR){Hair Removal};
    \node [block, below of = HR, node distance = 2.2cm](Seg){Segmentation};
     
	\begin{pgfonlayer}{background}
	\path (HR.west |- HR.north)+(-0.4,+1.2+\blockdist) node (a) {};
    \path (Seg.east |- Seg.south)+(+0.4,-2.0) node (b) {};          
    \path[fill=blue!10,rounded corners, draw=blue!30, dashed] (a) rectangle (b);
	\end{pgfonlayer}
	\path (HR.west |- HR.north) +(+1.4,0.8+\blockdist) node (pp) {\textbf{Pre-processing}};
	
    	% -- mapping block
    	\node[block, right of =HR, node distance = 4 cm](ml){Local};
    	\node[block, below of =ml, node distance = 2.2cm](mg){Global};
    \path(ml.north|-ml.east) +(0.1,+1.1\blockdist) node(p){};
        
	\begin{pgfonlayer}{background}
	\path (ml.west |- ml.north)+(-0.4,+1.2+\blockdist) node (a) {};
    \path (mg.east |- mg.south)+(+0.4,-2.0) node (b) {};          
    \path[fill=blue!10,rounded corners, draw=blue!30, dashed] (a) rectangle (b);
	\end{pgfonlayer}
	\path (ml.west |- ml.north) +(+1.4,0.8+\blockdist) node (mapping) {\textbf{Mapping}};
	
    % -- feature extraction block
	\node[block, right of = p, node distance = 3.9cm ](FE1){Texture};      
    \node[block, below of = FE1, node distance = 2.2cm ](FE3){Shape};
    \node[block, below of = FE3, node distance = 2.2cm ](FE2){Color};
    
    \begin{pgfonlayer}{background}
	\path (FE1.west |- FE1.north)+(-0.4,+0.1+\blockdist) node (a) {};
    \path (FE2.east |- FE2.south)+(+0.4,-0.95) node (b) {};          
    \path[fill=blue!10,rounded corners, draw=blue!30, dashed] (a) rectangle (b);
	\end{pgfonlayer}
	\path (FE1.west |- FE1.north) +(1.4,-0.2+\blockdist) node (FEB1) {\textbf{Feature}};
	\path (FE1.west |- FE1.north) +(+1.4,-0.6+\blockdist) node (FEB2) {\textbf{extraction}};
    
    % -- feature representation
    \path(FE1.north|-FE1.east) +(0.1,-1.1\blockdist) node(v){};
    \node[block, right of = v, node distance = 3.9cm](lr){Low}; 
    \node[block, below of = lr, node distance = 2.2cm](hr){High};
    
    \begin{pgfonlayer}{background}
	\path (lr.west |- lr.north)+(-0.4,+1.2+\blockdist) node (a) {};
    \path (hr.east |- hr.south)+(+0.4,-2.0) node (b) {};           
    \path[fill=blue!10,rounded corners, draw=blue!30, dashed] (a) rectangle (b);
	\end{pgfonlayer}
	\path (lr.west |- lr.north) +(1.4,0.9+\blockdist) node (FEB1) {\textbf{Feature}};
	\path (lr.west |- lr.north) +(+1.4,0.5+\blockdist) node (FEB2) {\textbf{representation}};
	
    % -- Balancing data block 
    %\path(dr.south|-dr.east)+(0.1,-1.1\blockdist) node(q){};
    \node[block, below of = hr, node distance = 7.1cm](Us){Under sampling};
    \node[block, below of = Us, node distance = 2.2cm](Os){Over sampling};
    
    	\begin{pgfonlayer}{background}
	\path (Us.west |- Us.north)+(-0.4,+1.2+\blockdist) node (a) {};
    \path (Os.east |- Os.south)+(+0.4,-2.0) node (b) {};              
    \path[fill=blue!10,rounded corners, draw=blue!30, dashed] (a) rectangle (b);
	\end{pgfonlayer}
	\path (Us.west |- Us.north) +(+1.4,0.8+\blockdist) node (balance) {\textbf{Balancing}};
	
	
    % -- classification block
    \node[block, below of = FE2, node distance = 6cm](SL){Single Learner};
    \node[block, below of = SL, node distance = 2.2cm](EL){Ensemble};
    
    \begin{pgfonlayer}{background}
	\path (SL.west |- SL.north)+(-0.4,+1.2+\blockdist) node (a) {};
    \path (EL.east |- EL.south)+(+0.4,-2.0) node (b) {};          
    \path[fill=blue!10,rounded corners, draw=blue!30, dashed] (a) rectangle (b);
	\end{pgfonlayer}
	\path (SL.west |- SL.north) +(+1.4,0.8+\blockdist) node (clsfy) {\textbf{Classification}};
    
	% --- The arrows 
	
	\path(FE3.north|- FE3.east)+(2.4, 0.0) node (f){}; 
	\path(f)+(-0.8,0) node (n) {};
	
	\path (hr.east |- hr.south)+(-1.4,-2.0) node (j) {};           
    \path (Us.west |- Us.north)+(+1.3,+1.2+\blockdist) node (k){}; 

	\path(f)+(-4.8,0) node(o){};
	\path(o)+(0.8,0)  node(r){};

	\path(o)+(-4.0,0) node(s){};
	\path(s)+(0.8,0)  node(t){};		

	\path(n)+(0,-9.0)  node (l) {};  
  	\path(f)+(0,-9.0)  node (m) {};  

	\path(s)+(-3.3,0) node(u){};
	
    \path [line] (input) -- (u);
    \path [line] (n) -- (f);
	\path [line] (j) -- (k);
	\path [line] (m) -- (l);
    \path [line] (o) -- (r);
    \path [line] (s) -- (t);

\end{tikzpicture}
\end{center}
\caption[The proposed automated framework]{Outline of the proposed algorithm for automatic recognition of melanoma.}
\label{fig:dermoscopyFramework}
\end{figure}