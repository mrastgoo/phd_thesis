\section{Malignant Melanoma} \label{sec:chp1sec3}
Although malignant melanoma accounts for less than 2\% of all skin cancer cases, it is the deadliest type and causes the vast majority of deaths \cite{CancerFactsFigures2014}.
The incidence of melanoma has increased in the past few decades currently reaching 132,000 melanoma cases per year according to the \ac{who}. 
%The \Ac{acs} also reported the estimated deaths of melanoma in 2014 as $9710$ individuals and new cases as $76,100$ individuals. 

Melanoma cancer is incurable in its advanced stages and the patient should go through surgery, possibly immunotherapy, chemotherapy, and/or radiation therapy. 
However, if it is diagnosed at an early stage, it is the most treatable kind of cancer~\cite{CancerFactsFigures2014,forsea2012melanoma}.
In fact, patient survival rate has increased significantly, over the past few decades, thanks to early diagnosis and treatment of melanoma in its early stages. 

The stages of melanoma are measured based on how the lesion has grown, including its invasion depth though the skin to nearby lymph nodes or other organs.
This factor is measured through physical exam, biopsies, and different imaging tests such as \ac{ct} or \ac{mri} ~\cite{CancerFactsFigures2014}.
Depending on the measurements obtained, melanoma skin cancers are divided into four types.

The first three types begin \textit{in situ}, meaning that they spread along the top layers of the skin and become invasive in the final stages, while the fourth is invasive from the start.
Invasive melanoma can be very dangerous since they are in the deeper layers of the skin and can spread faster to other body parts. The four types of cutaneous melanoma are listed in the following: 
	\begin{description}
	\item [Superficial spreading melanoma] 
	is the most common type and is the leading cause of cancer death in young adults. 
	Approximately 70\% of all melanoma diagnosis are counted as superficial spreading melanoma. 
	This type grows along the top layer of the skin and often occurs in a previously benign mole. 
	The location of this melanoma can be anywhere in the body, however, it is usually found on the trunk and back in male patients and on the legs and back in females patients.   
	\item [Acral lentiginous melanoma]
	accounts for less than 5\% of all melanoma diagnosis and is the most common type in dark skinned individuals. 
	This disease is usually located on the palms, soles of the feet and under the finger nails and often looks like a bruise or injury on the body.
	For this reason, the melanoma may be discovered later than other forms.  	
	\item [Letingo maligna melanoma]
	accounts for 5-10 \% of melanoma diagnosis. Letingo maligna melanoma arises from a pre-existing letingo rather than a mole and mostly occurs on the face of middle-aged and elderly individuals as a result of sun damage. 
If this cancer goes undiagnosed, being mistaken for a sun spot, it can spread to the deeper layers of the skin and endanger the patient's life. 
Cancer lesions of this type usually have a very irregular border and vary in shades of brown or black.
However, like other types of melanoma, they can be blue, red, gray or white. 
	\item [Nodular melanoma]
	accounts for 15-30\% of all melanoma diagnosis. 
	This is the most aggressive type due to the fact that it spreads more rapidly in depth and it is difficult to visualize the progression of the cancer.
Nodular melanoma is more common in males than in females. 
The lesion is usually darkly pigmented (blue-black) and is often found in pink or red. 
	\end{description}
	
%%% Local Variables: 
%%% mode: latex
%%% TeX-master: "../thesis"
%%% End: 