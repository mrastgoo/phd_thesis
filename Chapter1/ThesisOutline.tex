\section{Thesis Outline}
\label{sec:chp1sec7}
This thesis describes the research work that resulted in the development and validation of the Stokes polarimetry device and a classification frameworks for differentiation of melanoma using conventional cross-polarized and non-conventional image polarimetry techniques. 

Prior to developing a polarimetry system and a \ac{cad} framework, the basics and principles of polarization and classification frameworks are explored.
The basics of polarization and a history of polarimetry images is presented in \textbf{Chapter~\ref{chp:chapter4}} while \textbf{Chapter~\ref{chp:chapter2}} discuss the related machine learning and computer vision techniques related to the classification framework.
%Since classification is the base of our \ac{cad} systems, related machine learning and computer vision techniques are discussed in \textbf{Chapter~\ref{chp:chapter2}}. 

\textbf{Chapter~\ref{chp:chapter3}} is dedicated to dermoscopy modality. 
State of the art of \ac{cad} systems, our proposed framework, experiments and results obtained for dermoscopy modality are depicted in this chapter. 

\textbf{Chapter~\ref{chp:chapter5}} presents the framework developed for the \ac{pi} modality. 
The \Ac{pi} system and the developed classification framework along with the results obtained are presented in this chapter. 

Finally \textbf{Chapter~\ref{chp:chapter6}} concludes the thesis and presents avenues for future research.

%Additional information on the techniques used in the implementation of the PSL mapping and change detection pipelines can be found in the \textbf{Appendix}. %~\ref{sect:appendixA}}.
	
