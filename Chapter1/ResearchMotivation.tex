\section{Research Motivation}
\label{sec:chp1sec6}
Malignant melanoma is the deadliest type of skin cancer and accounts for the vast majority of skin cancer deaths~\cite{CancerFactsFigures2014}.
According to the latest reports, melanoma causes over 20,000 deaths annually in Europe~\cite{forsea2012melanoma}.
The \Ac{acs} also reported an estimated deaths of melanoma in 2014 as $9710$ individuals and new cases as $76,100$ individuals~\cite{CancerFactsFigures2014}.
Nevertheless, melanoma is the most treatable kind of cancer if diagnosed early.
Therefore, prevention and early diagnosis of melanoma lesions is crucial for the patients survival rate.

\acl{cad} systems have been proposed by the research community to assist dermatologists in early diagnosis of melanoma.
These systems are proposed to classify \acf{psls}.
Dermoscopy being the most common source of skin imaging, most of the \ac{cad} systems proposed are based on polarized dermoscopy (PD).
The \ac{pd} allows the visualization of the subsurface anatomic structure of the epidermis and papillary dermis.
Although \ac{pd} are used extensively by the dermatologist to document and analyze the \ac{psls}. 
The advantage and benefits of \acf{pi} beyond the cross-polarized filters, have not been fully explored in the field of skin cancer.

This work attempts to analyze more closely \ac{pi} in the field of skin cancer and intends to propose an automatic classification framework based on dermoscopy and image polarimetry.
For this purpose, a novel image polarimetry system able to provide the first three Stokes parameters is presented.
This system is implemented based on Stokes polarimetry since it can provide automatic, fast, accurate and less complex acquisition, in comparison to Mueller polarimetry. 

Towards providing a \ac{cad} system, different aspects of automated classification of \ac{psls} are extensively investigated and a \ac{cad} system based on dermoscopy images is proposed. 
Further using the proposed image polarimetry system, polarization properties of the \ac{psls} are analyzed and a \ac{cad} system based on new polarized features is presented.


