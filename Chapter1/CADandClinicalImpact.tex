\section{Automated Diagnosis of Melanoma}
\Ac{cad} or \ac{cds} systems are proposed to provide automated diagnosis of melanoma lesions. 
These systems are intended to reproduce the decision of the dermatologists when observing images of \ac{psls}. 
Automated diagnosis of melanoma was proposed to assist dermatologists and increase the sensitivity and specificity of melanoma recognition in their early stages as well as to reduce unnecessary excisions.
From a computer vision and pattern recognition point of view, \ac{cad} systems for melanoma intend to classify and differentiate melanoma lesions from others.
In general, image processing techniques are used to locate and delineate the lesions and extract the image parameters (features) which coincide with the dermatologist's point of view and dermatological features. 
The extracted parameters are further used with machine learning tools to perform a diagnosis (classification). 
Such systems are being developed for various imaging modalities~\cite{Marchesini2002,Vestergaard2008,korotkov2012computerized}. 
However, dermoscopy being the most conventional imaging technique, most \ac{cad} systems are dedicated to this modality. 
%This research is aimed to provide an automate classification framework of melanoma lesions based on conventional dermoscopy and non-conventional \ac{pi}.

This research is aimed at analyzing the effects of polarized illumination beyond that of usual dermoscopes, for the detection of melanoma lesions and providing a \ac{cad} system for an automated classification of melanoma lesions based on conventional dermpscopes and image polarimetry.
%In our research, we considered automated classification of melanoma based on conventional dermoscopy, and non-conventional \ac{msi} and \ac{pi} imaging.  
The general steps of a \ac{cad} system are extensively discussed in Chap.\,\ref{chp:chapter2}.

\subsubsection{Clinical impact}
Numerous research projects have been dedicated to the development of a \ac{cad} or \ac{cds} system and studies show that their performance is sufficient under experimental conditions~\cite{Fruhauf2012}. 
The proposed systems can assist dermatologists, however, their practical value is still unclear and they cannot be recommended as a sole determinant of malignancy of a lesion.
Even though most patients would accept a computerized analysis of a melanoma, the \ac{cad} systems proposed cannot function alone due to their tendency to over diagnose benign melanocytic and non-melanocytic lesions~\cite{Fruhauf2012}.
Day and Barbour~\cite{Day2001} listed two main shortcomings for general approaches which are adapted to develop a \ac{cad} system for recognition of a melanoma:
\begin{enumerate}
\item A \ac{cad} system is expected to reproduce the decision of pathologists (a binary result like ``melanoma/non-melanoma lesion'') with only the input used by dermatologists: clinical or dermoscopic images;
\item Histopathological data are not available for all lesions, only for those considered suspicious by dermatologists.
\end{enumerate}
\noindent
The items listed refer to the lack of sufficient information and interaction with dermatologists used in the proposed method.
These items refer to the fact that the proposed methods lack sufficient information and interaction with dermatologists.
This fact was highlighted by Dreiseitl~et al.\,\cite{Dreiseitl2005} as well. 
The authors mentioned that current systems are designed to work ``in parallel with and not in support of'' physicians, thus only a few systems have been used in clinical routines.
In this regard, an ideal \ac{cad} or \ac{cds} system for melanoma recognition should be able to reproduce the dematologists's decision with extensive information regarding the reason and ground for that decisions~\cite{Dreiseitl2005}.
