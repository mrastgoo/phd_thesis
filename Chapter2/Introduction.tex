\section{Introduction} \label{sec:chp2-sec1}
Machine learning is refers to a set of methods that can automatically detect patterns in data and use them in decision making and future data prediction~\cite{murphy2012machine}. 
Machine learning techniques are usually divided into two groups: descriptive (or unsupervised) and predictive (or supervised) learning.

Unsupervised learning intends to find ``interesting structures'' in a given data without any additional information about the expected output. 
These techniques formulate their problems as unconditional density estimation and multivariate probability models. 
Clustering approaches and dimension reduction methods, \acf{pca}, and graph structures are two examples of unsupervised learning.   

Supervised learning intends to find a mapping $f(.)$ which relates a set of inputs $x$ to a set of outputs $y$. 
The learning is comprehended using a set of $N$ samples and their labels $D = \{(x_{i}, y_{i})\}^{N}_{i = 1}$, called the training set.
The samples $x_{i}$ can vary from 2-D points in Cartesian coordinates to more complex forms such as images, sentences, time series, graphs, etc.
Similarly, the labels $y_{i}$ can be represented by any structure, however, in most methods they are considered to be either categorical or numerical. 
Supervised learning is known as a classification problem when $y_{i}$ is categorical; and when $y_{i}$ is a real value the problem is referred to as regression~\cite{murphy2012machine}. 

In this research, we are interested in supervised learning, and the classification problem in particular. 
Classification aims to map a set of inputs to a set of outputs, where the outputs are divided into different classes, $y \in \{1,...,C\}$.
%{\color{green}As previously mentioned classification aims to maps a set of input $x$ to a set of output $y$, where $y \in \{1,...,C\}$ and C is the number of classes.} 
If $C = 2$, it is a binary classification problem, while $C > 2$ makes it a multiclass classification. 
This project aims to separate melanoma from all other pigmented skin lesions, thus addressing a binary classification problem. 

Classification has numerous applications in real life.
Some of the most common and challenging areas include email spam filtering, data mining, face detection and recognition, document classification, and the problem of cancer detection among others.
In order to solve a classification problem, a framework consisting of several generic steps is required.
Figure~\ref{fig:GCF} shows such a general framework whose steps are described in the following sections of this chapter. 
Due to the broad studies of \ac{cad} systems of melanoma using dermoscopic images, in comparison to Stokes polarimetry, each step covers the state of the art related to conventional dermoscopic approaches.

%\tikzstyle{block} = [rectangle, draw, fill= blue!20,
%   text width=7em, text centered, rounded corners, minimum height=4em , minimum width = 7em]
%\tikzstyle{line} = [draw, -latex']
%\tikzstyle{block2} = [rectangle, draw, fill=gray!20,
%    text width=7em, text centered, rounded corners, minimum height=4em, minimum width = 7em]
%\tikzstyle{block3} = [rectangle, draw, fill=blue!40,
%    text width=7em, text centered, rounded corners, minimum height=3em , minimum width = 7em]
%\def\blockdist{1}
%\def\edgedist{1.5}
    \tikzstyle{block} = [rectangle, draw, fill=blue!30,text = black,
    text width=6em, text centered, rounded corners, minimum height=4em , minimum width = 7em]
    \tikzstyle{line}=[draw, -latex']
	\tikzstyle{block2} = [rectangle, draw, fill=white!20,
    text width=6em, text centered, rounded corners, minimum height=4em, minimum width = 7em]
    \tikzstyle{block3} = [rectangle, draw, fill=blue!30, text = black,
    text width=7em, text centered, rounded corners, minimum height=4em , minimum width = 7em]
\def\blockdist{1}
\def\edgedist{1.5}
\begin{figure}
\centering   
\begin{tikzpicture}[node distance = 1cm,scale=0.8, every node/.style={scale=0.8}]
    % Place nodes
	\node[block2] (input) {Input Data}; 
	\node[block, right of= input , node distance = 3cm](pp){Pre-processing};
	\node[block, right of = pp, node distance = 3 cm](fe){Feature extraction}; 
	\node[block, right of = fe, node distance = 3 cm](fr){Feature representation}; 
	\node[block, right of = fr, node distance = 3 cm](db){Data balancing}; 
	\node[block, right of = db, node distance = 3 cm](clas){Classification}; 
	
	% --- The arrows 
	
	%\path(pp)+(-0.8,0) node (n) {};
	\path [line] (input) -- (pp);
    \path [line] (pp) -- (fe);
    \path [line] (fe) -- (fr);
    \path [line] (fr) -- (db);
    \path [line] (db) -- (clas);
   

\end{tikzpicture}
\caption[General classification framework]{General classification framework.}
\label{fig:GCF}
\end{figure}
 
