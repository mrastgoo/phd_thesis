\section{Feature Extraction} \label{sec:chp2-sec3}
Feature extraction refers to the process of gathering a set of characteristics from samples that meaningfully and efficiently describe the most important information needed for further data analysis and classification.
The need for feature generation in the image processing field originated from our inability to use raw data. 
Even for a small $64 \times 64$ image, having all the pixels as features results in $4096$ feature dimensions which is too much for most classifiers~\cite{Theodoridis2006327}.
Thus, new features need to be generated from sample images.

Numerous approaches have been proposed for feature extraction in the literature.  
Based on the characteristics of the desired features, these approaches can be divided into four main categories; shape, color, texture, and edge.
Furthermore, they can be categorized into two general categories of pixel-wise and region-wise features. 
The former means that features are extracted at each pixel, and the later refers to a descriptor describing a region (i.e. histogram, percentile and moment are region-based). 
The edge features created by convolving the image with an operator such as sobel and prewitt, belong to pixel-wise features~\cite{lemaitre2015computer}. 
Some texture features such as the Gabor filter~\cite{gabor1946theory,daugman1985uncertainty}, wavelet, the \acf{glcm}~\cite{haralick1973textural}, fractal analysis~\cite{benassi1998identifying} and the \ac{lbp}-map extract features for each pixel.
However, others, such as \acf{hog}~\cite{dalal2005histograms}, histogram of \acf{lbp}-maps~\cite{ojala1996comparative}, the \acf{sift}~\cite{lowe2004distinctive}, and \acf{surf}~\cite{bay2006surf}, generate descriptors which are the most representative of a region.
Color features, such as statistics and color histograms, and shape features, such as asymmetry and thinness ratio, belong to this category as well.  

Among the aforementioned features, shape and color features have been widely adapted for classification of melanoma using dermoscopy images. 
These features simulate the most, characteristics of diagnostic rules such as ``ABCD''.
An extensive review of the feature extraction methods applied, using dermoscopic images is discussed in the following chapter (see Chapt.~\ref{chp:chapter3}, Sect.~\ref{chp3-subsec3}).

%{\color{red} check the pattern recognition book, shape features are region-wise features, thinness ratio, asymmetry, ... } 





%%% Local Variables: 
%%% mode: latex
%%% TeX-master: "../thesis"
%%% End: 
