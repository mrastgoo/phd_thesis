%\section{Introduction}\label{sec:chp4-sec1}
Light was thought as being non-scalar for the first time by Christian Huygens when he observed the propagation of light through crystals~\cite{goldstein2003polarized}.
Through observations it appeared that light had ``sides'', in the words of Newton.
This new vectorial nature of light was called polarization.
Later, Frensel and Arago discovered that light consisted of only two transverse components and the perpendicular components were assumed to propagate in the $z$ direction.
%The light was suggested for the first time to be non scalar by Christian Huygens when he observed the propagation of light through crystals. It appeared that light had "sides" in the words of Newton. The new vectorial nature of light is called polarization. Later Fresnel and Arago discovered that light consisted only of two transverse components. The components are perpendicular to each other and are assumed to propagate in $z$ direction. 

The concept of polarization arises from the transverse and vector nature of electromagnetic radiation.
This concept describes the resultant pattern of an electric field vector ($E(r,t)$) as a function of time ($t$) at a fixed location in space ($r$).
The classical theory of polarization and the nature of light are evidence that polarization is another fundamental property of light besides coherent, frequency and intensity.

%This chapter is organized such as, first polarized properties of light and the mathematical representation of these properties are presented, then the polarized properties of the medium and it's mathematical formulations are explained and finally the related works on image polarimetry systems are discussed.
In this chapter, the polarized properties of light and the mathematical representation of these properties are explained first.
This section focuses on the mathematical formulations used in dermopolarimetry. 
Later the polarized properties of the medium and it's mathematical formulations through the Stokes representation are presented and, finally, the related work on image polarimetry systems are explained.



%Finally we discuss the image polarimetry systems proposed by the research community.
% This chapter is organized such as, first the basic and principals of polarization are discussed, then the image polarimetry and related works to this filed are presented.