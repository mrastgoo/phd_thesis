%\section{Stokes Parameters}\label{sec:chp4-sec3}
%Characteristics of the polarized light are explained with different formulations, such as Stokes parameters, Mueller matrix, Jones matrix, and Pincor\'e sphere.
%The full explanation of all these formulations are beyond our scope and we limited our explanations to Stokes and Mueller characteristics.


\section{Stokes Parameters}\label{sec:chp4-sec3}
In 1852, Sir George Gabriel Stokes discovered that polarization behavior could be represented in terms of observables.
Sir.~Stokes in an attempt to mathematically characterize un-polarized light, defined un-polarized light as light whose intensity is unaffected when a polarizer is rotated or while a retarder of any retardance value is present. 
In this characterization Sir~Stokes defined four measurable quantities known as  the Stokes polarization parameters. (See Eq.\ref{Eq:Stokeseq}).
\begin{equation}\label{Eq:Stokeseq}
\small
	(E_{0x}^{2}+ E_{0y}^{2})^{2} - (E_{0x}^{2}- E_{0y}^{2})^{2}- (2E_{0x}E_{0y}\cos\delta)^{2} =  (2E_{0x}E_{0y}\sin\delta)^{2}~,
\end{equation}
The first parameter expresses the total intensity of the optical field in terms of the total amount of horizontal and vertical linear polarization.
The second and third parameters describe the amount of linearly polarized light and the fourth parameter describes the amount of left or right circularly polarized light.
The total \acf{dop}, the \acf{dolp} and the \acf{docp} are calculated using Stokes parameters.
Equation.~\ref{Eq:stokevector} represents the Stokes vector with the parameters represented first in terms of amplitudes of the transverse components ($E_{0x}, E_{0y}$) and  $\delta$ angle, then using different states of polarization.
Here $I_{H}$ and $I_{V}$ stand for linear horizontal and vertical polarized, respectively and $I_{P}$, $I_{M}$, $I_{L}$, $I_{R}$ represent linear $+45^{\circ}$, linear $-45^{\circ}$, left circular and right circular polarized light, respectively. 

\begin{equation}\label{Eq:stokevector}
\small
S = 
	\begin{bmatrix}
	S_{0}\\
	S_{1}\\
	S_{2}\\	
	S_{3}\\
	\end{bmatrix} 
	= 
	\begin{bmatrix}
	I\\
	Q\\
	U\\	
	V\\
	\end{bmatrix} 
	= 1/2
	\begin{bmatrix}
	E_{0x}^{2} + E_{0y}^{2} \\
	E_{0x}^{2} - E_{0y}^{2}\\
	2E_{0x}E_{0y}\cos\delta\\	
	2E_{0x}E_{0y}\sin\delta\\
	\end{bmatrix} 
	= 
	\begin{bmatrix}
	I_{H}+ I_{V}\\
	I_{H}- I_{V}\\
	I_{P}- I_{M}\\
	I_{L}- I_{R}\\
	\end{bmatrix}~.	
\end{equation}

The four Stokes parameters are real quantities in terms of intensities and, for any state of polarized light, they always satisfy the relation shown in Eq.~\ref{Eq:Stokesineq}.
The equality sign applies when completely polarized light exists and inequality applies when partially polarized light or un-polarized light exists.
\begin{equation}\label{Eq:Stokesineq}
\small
	S_{0}^{2}\geqslant S_{1}^{2}+ S_{2}^{2}+S_{3}^{2}~.
\end{equation}

As mentioned above, the first element of the Stokes vector represents the total intensity of light.
The \ac{dop} of light is defined as the ratio of polarized intensity per total intensity.
In a similar manner, \ac{dolp} and \ac{docp} are calculated.
\begin{subequations}\label{Eq:DP}
\small
	\begin{align}
	\ac{dop}  & =  \rho =\frac{I_{pol}}{I_{tot}} = \frac{(Q^2 + U^2 + V^2)^{1/2}}{I}~, \\
	\ac{dolp} & = \rho_{l}= \frac{(Q^2 + U^2 )^{1/2}}{I}~, \\
	\ac{docp} & = \rho_{c} = \frac{V}{I}~.
	\end{align}
\end{subequations}

With respect to Eq.~\ref{eq:eq8}, the relation between rotation and the ellipticity angle of the polarized ellipse and Stokes parameters are defined by: 

\begin{equation}\label{Eq:ellipparStokes}
\small
	\sin 2\chi = S_{3}/S_{0}   \hspace{1.5cm} \tan 2\psi = S_{2}/S_{1}~.
\end{equation} 
\noindent With regard to Eq.~\ref{Eq:ellipparStokes}, the Stokes representation in terms of rotation, the ellipticity angle and $\alpha$ is given by: 

\begin{equation}\label{Eq:nrStokesvec}
\small
S = 	 S_{0}
	\begin{bmatrix}
	1 \\
	\cos 2\chi \cos 2\psi\\
	\cos 2\chi \sin 2\psi\\	
	\sin 2\chi
	\end{bmatrix} 
	= I_{0}
	\begin{bmatrix}
	1\\
	\cos 2\alpha \\
	\cos 2\alpha \cos\delta\\	
	\sin 2\alpha \sin\delta
	\end{bmatrix}~.
\end{equation}

Table~\ref{tab:chp4T1} illustrates polarization states and their relative Stokes vector.
\begin{table}
\small
\begin{center}
\caption{Stokes vector for polarization states.}
\resizebox{1.0\linewidth}{!}{
\begin{tabular}{c c c c c c c c }
    \hline
   & & & & & & & \\[-2.5ex] 
   Polarization states & H & V & $+45^{\circ}$ & $-45^{\circ}$ & R & L & Elliptical\\
   & & & & & & &  \\[-2.5ex] 
   \hline
   &   & & & & & & \\[-1.5ex]
   Stokes Vector  &  &   &   &  &  & & \\
   $\begin{bmatrix}
    S_{0}\\ S_{1}\\ S_{2}\\ S_{3}    
   \end{bmatrix}$ &    
   $\begin{bmatrix}
   1\\ 1\\ 0\\ 0    
   \end{bmatrix}$& 
   $\begin{bmatrix}
   1\\  -1\\  0\\  0    
   \end{bmatrix}$& 
   $\begin{bmatrix}
   1\\ 0\\ 1\\ 0    
   \end{bmatrix}$& 
   $\begin{bmatrix}
   1\\ 0\\ -1\\ 0    
   \end{bmatrix}$&  
   $\begin{bmatrix}
   1\\ 0\\ 0\\  1   
   \end{bmatrix}$&    
   $\begin{bmatrix}
   1\\ 0\\ 0\\ -1    
   \end{bmatrix}$& 
   $\begin{bmatrix}
	1 \\
	\cos 2\chi \cos 2\psi\\
	\cos 2\chi \sin 2\psi\\	
	\sin 2\chi 
   \end{bmatrix}$\\
    &  &  &  &  &  &  & \\
    \hline   
  \end{tabular}
  }
  \label{tab:chp4T1}
  \end{center}
\end{table}
There are different ways of measuring Stokes parameters. 
A classic way of measuring these parameters is by passing an optical beam through two optical elements, known as the retarder and the polarizer.
The retarder is an element that changes the phase between two transverse components of the light, and the polarizer is an element that changes the amplitude of the transverse components.
The incident light passes through the retarder which then advances the phase of the $x$ component ($E_{x}$) by  $\phi/2$ and retards the phase of the $y$ component by $\phi/2$. 
The beam merging from the retarder (the phase shifting element) passes through the polarizer next.
The polarizer has the ability to transmit the optical field through only one axes called the transmission axis. 
For instance, if the transmission axis of the polarizer is at $\theta$, only the $E'_{x}$ and $E'_{y}$ components in this direction pass through.

Using this set-up, the first three Stokes parameters are measured by removing the retarder and rotating the transmission axis of the polarizer to the angles $\theta = 0^{\circ}, \theta = +45^{\circ}$ and $\theta = +90^{\circ}$, respectively.
For measuring the final parameter, a retarder is added to the set-up with the angle of $\phi = 90^{\circ}$ (quarter-wave retarder), while the transmission axis of the polarizer is set to $\theta = +45^{\circ}$.
Equation.~\ref{Eq:cStokesmeas} represents how this set-up can be used to measure the Stokes parameters.
$I(\theta, \phi)$ shows the intensity produced by the retarder with a $\phi$ phase shift angle and the polarizer with its transmission axis at $\theta$.

 \begin{subequations}\label{Eq:cStokesmeas}
 \small
% \begin{center} 
 	\begin{align}
 	S_{0} & =  I(0, 0) + I(90,0)~, \\
 	S_{1} & =  I(0,0) - I(90,0)~, \\
 	S_{2} & = 2I(45,0)-I(0,0)-I(90,0)~,\\
 	S_{3} & = 2I(45,90) - I(0,0) -I(90,0)~. 
	\end{align} 
 %\end{center}
 \end{subequations}



